\documentclass[oneside,openany,letterpaper,12pt]{book}
\pagestyle{plain}
\usepackage{geometry}
\geometry{top=2cm,bottom=2cm,left=1.5cm,right=2cm} % other options--margin,marginparwidth, landscape, a4paper
\usepackage[utf8]{inputenc}
\usepackage[T1]{fontenc}
\usepackage[spanish,english]{babel}
\usepackage{mathptmx}  % Letra Times New Roman; otros tipos de letra (e.g. lmodern)
\usepackage{microtype} % mejor tipografía para pdflatex
\usepackage{csquotes} % mejores comillas
\usepackage[shortlabels]{enumitem} % añade mejor control para la configuración de listas
\usepackage{booktabs} % mejores reglas horizontales para tablas
\usepackage{array} % permite más control del tamaño de celdas en una tabla
\usepackage{graphicx} % permite añadir y manejar imágenes en el documento
\graphicspath{ {./imagenes/} } % localidad del archivo de imagenes
\usepackage[style=authoryear-comp,sorting=nyt,natbib=true,backend=biber]{biblatex}
\addbibresource{referencias.bib} % nombre del archivo con las referencias en formato bib(la)Tex
\usepackage{setspace} % permite configurar los espacios de interlineado
\doublespacing % configura el documento en espacio doble otros -- 
\usepackage{blindtext}
\usepackage{lipsum}
\usepackage[colorlinks=true, allcolors=blue]{hyperref}
\usepackage{titlesec}
\usepackage{color}
\definecolor{gray75}{gray}{0.75}
\newcommand{\hspdiez}{\hspace{10pt}}
\titleformat{\chapter}[hang]{\Large\bfseries}{\thechapter\hspdiez}{0pt}{\Large\bfseries}
\titleformat{\section} {\normalfont\bfseries} {\thesection.}{0.5em}{\normalfont\bfseries}


\usepackage{pdflscape} % cambia la orientación de la página deseada
%\usepackage{parskip} % maneja la identación de los párrafos, mirar mejor
%\usepackage{url}

%\usepackage{todonotes} % add notes of things to add to the manuscript

% % handle line spacing
%\usepackage{verbatim} % add multi line comments
